\documentclass{article}
\usepackage{polski}
\usepackage{geometry}

\geometry{
    a4paper,
    total={190mm,270mm},
    left=10mm,
    top=10mm,
}

\begin{document}
\section{Tytuł projektu i przypomnienie problemu}
Projekt \textit{Wika Gotuje} jest rozwiązaniem problemu ludzi, którzy chcą swobodnie dzielić się swoimi przepisami kulinarnymi bez konieczności zakładadnia blogów lub dedykowanych 
witryn internetowych. Nasz projekt, również ma być odpowiedzią dla osób, które poszukują receptur dostosowanych do ich prefencji, a także sprawdzonych przez innych pasjonatów gotowania.

\section{Opis użytkowników i podział na klasy}
\subsection{Przeciętny użytkownik szukający przepisu}
  \begin{itemize}
    \item Przeciętna osoba z dostępem do internetu
    \item Szeroki zakres wiekowy (10-90 lat), sprawność fizyczna na poziomie umożliwiającym samodzielne przyrządzanie potraw
    \item Podstawowe doświadczenie z korzystania ogólnie pojętego internetu
    \item Żadne wcześniejsze przeszkolenie z użytkowania witryny nie jest konieczne; użytkownik po prostu wchodzi na stronę internetową
    \item Wymagana jest znajomość czytania (i rozumienia) w języku polskim
    \item Znajomość swoich preferencji i umiejętności, a także świadomość dostępności sprzętu kuchennego i danych składników
    \item Zainteresowani danym przepisem często kierują się opiniami innych użytkoników, którzy zdecydowali się skorzystać z danej receptury
  \end{itemize}
\subsection{Użytkownik chcący udostępnić swój przepis innym}
  \begin{itemize}
    \item Osoba z pewnym doświadczeniem kulinarnym mająca dostęp do internetu
    \item Szeroki zakres wiekowy (10-90 lat), sprawność fizyczna na poziomie umożliwiającym korzystanie z komputera/telefonu
    \item Podstawowe doświadczenie z korzystania ogólnie pojętego internetu
    \item Żadne wcześniejsze przeszkolenie z użytkownia strony nie jest konieczne; użytkownik musi jednak założyć wcześniej swoje konto w serwisie
    \item Wymagana jest znajomość czytania, pisania i jasnego przekazywania informacji innym
    \item Wymagane jest doświadczenie w przygotowywaniu potrawy własnego przepisu
    \item Osoby odpowiedzialne za jakiś przepis często udzielają odpowiedzi na różne pytania, np. dotyczące możliwości zastąpienia jednego składnika innym
  \end{itemize}
\subsection{Moderator portalu}
  \begin{itemize}    
    \item Osoba powołana na swoje stanowisko, posiadająca doświdczenie kulinarne
    \item Zakres wiekowy (18-50 lat), sprawność fizyczna na poziomie pozwalającym regularne przeglądanie nowo dodanych przepisów, a także komentarzy
    \item Podstawowe doświdczenie z korzystania internetu/komputera/telefonu
    \item Wymagane jest wcześniejsze przeszkolenie z obsługi platformy z perspektywy moderatora
    \item Spodziewana jest umiejętność odróżniania nieprawdziwych/niepoprawnych treści (przepisy, komentarze) od poprawnych 
    \item Moderator może komunikować się z osobą odpowiedzialną za dany przepis z prośbą o dokonanie pewnych modyfikacji (uzupełnienie potrzebnych narzędzi, poprawne ilości danego produktu)
  \end{itemize}
\newpage
\section{Opis zadań}
\subsection{Znalezienie przepisu zgodnego z własnymi preferencjami}
  \begin{itemize}
    \item \textbf{Warunki wstępne:}
      \begin{itemize}
        \item posiadanie informacji na temat preferowanych smaków, ulubionych kuchni lub rodzajów dań; pomaga to w wybraniu konkretnego przepisu
        \item ocena swoich własnych umiejętności kulinarnych (przygotowanie bardzo zaawansowanych dań przez kucharza amatora jest ryzykowne) 
        \item zastanowienie się ile wolnego czasu mamy na przygotowanie posiłku (duży czas realizacji danego przepisu może go skreślić)
      \end{itemize}
    \item \textbf{Potencjalne przeszkody:}
      \begin{itemize}
        \item brak zróżnicowanych propozycji
        \item podane propozycje wymagają zbyt wysokiego poziomu umiejętności lub niedostępnych w danym rejonie składników itd.
      \end{itemize}
    \item \textbf{Podzadania:}
      \begin{itemize}
        \item filtrowanie przepisów pod względem parametrów takich jak: poziom zaawansowania, rodzaj kuchni, rodzaj dania (zadanie opcjonalne)
      \end{itemize}
  \end{itemize}
\subsection{Udostępnianie swojego przepisu innym użytkownikom platformy}
  \begin{itemize}
    \item \textbf{Warunki wstępne:}
      \begin{itemize}
        \item posiadanie kompletnego przepisu, zawierającego dokładne informacje dotyczące potrzebnych składników i ich ilości, kroków przygotowywania oraz informacji o kategoriach (np. czy danie główne, kuchnia włoska itd.)
        \item zrobienie zdjęcia gotowej potrawy (dodanie chociaż jednego zdjęcia jest wymagane)
        \item użytkownik musi posiadać własne konto w serwisie
      \end{itemize}
    \item \textbf{Potencjalne przeszkody:}
      \begin{itemize}
        \item brak zdjęć potrawy (dany przepis zrealizowaliśmy dawno temu, ale nie zrobiliśmy zdjęcie gotowego posiłku)
        \item dodawane zdjęcie ma zbyt duży rozmiar
        \item nie posiadamy wszystkich potrzebnych danych, aby dodać nowy przepis (np. nie znamy liczby porcji gotowego dania)
      \end{itemize}
  \end{itemize}
\subsection{Wydrukowanie przepisu}
  \begin{itemize}
    \item \textbf{Warunki wstępne:}
      \begin{itemize}
        \item znalezienie przepisu, który będziemy chcieli wydrukować
        \item upewnienie się, że nasze urządzenie ma połączenie z drukarką, a sama drukarka jest gotowa do druku
      \end{itemize}
    \item \textbf{Potencjalne przeszkody:}
      \begin{itemize}
        \item nasz zestaw do drukowania nie działa poprawnie
        \item wydrukowany przepis jest źle dopasowany do strony (np. nie mieści się na jednej stronie, co utrudnia korzystanie z niego podczas gotowania)
      \end{itemize}
  \end{itemize}

\end{document}
