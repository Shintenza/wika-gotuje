\documentclass{article}
\usepackage{graphicx} % Required for inserting images
\usepackage{polski}
\usepackage{geometry}

\geometry{
    a4paper,
    total={190mm,277mm},
    left=10mm,
    top=15mm,
}

\title{\fontsize{20}{22}\selectfont Projekt zespołowy\\ "Wika Gotuje" \\ część 4\\Testowanie prototypu}

\begin{document}
\section{Krótki opis projektu}
Projekt "WikaGotuje" jest rozwiązaniem dla użytkowników chcących dzielić się swoją pasją do gotowania z innymi. Wiele obecnie istniejących witryn z przepisami nie oferuje możliwości
dodawania własnych receptur, a ich interfejsy w dużej mierze są przestarzałe, niejednokrotnie utrudniające znalezienie konkretnego przepisu (np. ze względu na ubogie możliwości
filtrowania lub braku mobilnej wersji witryny). Brakuje im także jakichkolwiek funkcji ułatwiających życie, jak zapisywanie polubionych przepisów czy np. generowanie
przepisu w wersji do druku. Odpowiedzią na wymienione problemy ma być właśnie nasz projekt.\newline

Reasumując "WikaGotuje" ma umożliwiać:
\begin{itemize}
  \item Znalezienie przepisu zgodnego z własnymi preferencjami (wykorzystując różne filtry takie jak trudność przygotowania czy region pochodzenia)
  \item Dodanie własnego przepisu
  \item Wydrukowanie znalezionego przepisu/zapisanie w formie PDF
  \item Zapisywanie polubionych przepisów, aby potem znaleźć je łatwiej
  \item Ocenianie i komentowanie przepisów
  \item Obserwowanie osoby, której przepisy nam się podobają (otrzymywanie powiadomień gdy ta osoba coś doda)
\end{itemize}

\section{Zadania dla użytkowników}
\subsection{Zadanie 1}
\textbf{Treść:} Wydrukuj przepis/pobierz do PDF\newline
\textbf{Informacje dodatkowe:} Sam wybierasz przepis, który chcesz wydrukować  

\subsection{Zadanie 2}
\textbf{Treść:} Dodaj przepis\newline
\textbf{Informacje dodatkowe:} Szczegóły przepisu
\begin{itemize}
  \item danie z mięsem, pochodzące z Gruzji o nazwie \textit{Chinkali}
  \item możliwe do przygotowania w 2 godziny; otrzymamy mniej więcej 4 porcje gotowej potrawy
  \item składniki jest łatwo dostać w sklepach, a sam proces przygotowywania nie jest skomplikowany
\end{itemize}

\subsection{Zadanie 3}
\textbf{Treść:} Skomentuj przepis\newline
\textbf{Informacje dodatkowe:} Chcemy zarówno zostawić opinię tekstową, jak i ocenę w skali od 1 do 5


\subsection{Zadanie 4}
\textbf{Treść:} Polub przepis\newline
\textbf{Informacje dodatkowe:} sprawdź czy przepis faktycznie został dodany do twoich polubionych 

\textbf{Link do strony:} https://wikagotuje.scnstr.dev/

\end{document}
