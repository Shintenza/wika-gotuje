\documentclass{article}
\usepackage{graphicx} % Required for inserting images
\usepackage{polski}
\usepackage{geometry}
\geometry{
    a4paper,
    total={190mm,277mm},
    left=10mm,
    top=10mm,
}

\newenvironment{itemize.zip}
{ \begin{itemize}
    \setlength{\itemsep}{0pt}
    \setlength{\parskip}{0pt}
    \setlength{\parsep}{0pt}     }
{ \end{itemize}                  }

\title{\fontsize{20}{22}\selectfont Projekt zespołowy\\ część 1\\Strona internetowa z przepisami kulinarnymi\\"Wika gotuje"}
\date{}

\begin{document}

\section{Opis problemu}
\noindent
Użytkownicy często napotykają na problem w znalezieniu przepisów kulinarnych, z których mogą przygotować posiłek dostosowany do ich preferencji. Problematyczne jest również dzielenie
się własnymi odkryciami gastronomicznymi; obecnie aby dotrzeć do innych ze swoimi przepisami konieczne jest zakładanie swojej własnej strony lub bloga. Wiele obecnie istniejących
rozwiązań nie posiada także systemu ocen danych przepisów, nie oferuje interfejsu użytkownika, który jest przyjazny do używania podczas gotowania lub nie posiada mechanizmu zapisywania
ulubionych receptur. Projekt "Wika gotuje" ma być rozwiązaniem powyższych problemów po przez stworzenie przyjaznej i prostej w obsłudze strony internetowej.
% "Wika gotuje" rozwiązuje problem związany z brakiem możliwości udostępniania swoich przepisów dla pasjonatów gotowania. Użytkownicy często napotykają trudności w znalezieniu prostych przepisów do gotowania, a także inspiracji kulinarnych, dzieleniu się swoimi pomysłami lub organizowaniu swoich przepisów. 
% \newline
% Istniejące platformy mogą zawierać zbyt wyszukane przepisy lub posiadać przestarzały, nieresponsywny interfejs.

\subsection{Cele użytkownika}
\begin{itemize.zip}
    \item \textbf{Dzielenie się przepisami kulinarnymi:} użytkownicy chcą dzielić się swoimi przepisami z innymi, bez konieczności zakładania swoich stron internetowych lub blogów.
    % \item \textbf{Odkrywanie nowych dań:} poszukiwanie nowych, ekscytujących przepisów i kuchni, które zaspokoją ich gust.
    \item \textbf{Odkrywanie nowych dań:} poszukiwanie nowych, dostosowanych do własnych preferencji przepisów, które będą odpowiedne zarówno dla studenta, jak i doświadczonego kucharza.
\end{itemize.zip}

\subsection{Przeszkody}
\begin{itemize.zip}
    \item \textbf{Brak odpowiedniego sprzętu kuchennego}: brak stosownego wyposażenia kuchennego (takiego jak piekarnik czy robot kuchenny) przeważnie skreśla dany przepis
    \item \textbf{Budżet i dostępność składników}: wysoki stopień wyszukania danego składnika, a także jego cena może być poważną przeszkodą
\end{itemize.zip}


\section{Używane technologie}
      Next.js - frontend i backend framework,
      Sass - arkusze styli,
      Netlify - hosting,
      MongoDB - baza danych.

\section{Docelowy użytkownik}
"Wika gotuje" jest skierowana do wszystkich pasjonatów gotowania niezależnie od poziomu doświadczenia kulinarnego. Różnorodność użytkowników obejmuje zarówno kuchcików-amatorów, jak 
i doświadczonych szefów kuchni. Również mniejsze doświadczenie technologiczne i ogólne obycie z internetem nie przekreśla docelowego użytkownika z użytkowania naszej platformy.
% \newline
% Docelowy użytkownik powinien mieć też możliwość łatwego filtrowania, zapisywania i sortowania przepisów, aby łatwo znaleźć preferowane receptury.

% Platforma  powinna być filtrowana pod względem preferencji diety i poziomów umiejętności, aby zapewnić spersonalizowane doświadczenia.


\section{Rozwiązanie}
Stworzenie aplikacji internetowej umożliwiającej:
\begin{itemize.zip}
    \item łatwe utworzenie profilu użytkownika (z użyciem platform takich jak facebook, google itd.) i zarządzanie nim (ustawienie poziomu zaawansowania w gotowaniu, opis profilu),
    \item dodawanie i ocenianie przepisów kulinarnych,
    \item możliwość pobierania przepisu w wersji PDF do druku,
    \item przyjazne użytkowanie w trakcie gotowania (możliwość powiększenia czcionki, zmiana motywu strony na jasny/ciemny),
    \item filtrowanie i sortowanie przepisów po kategorii/poziomie trudności itd.,
    \item dodawanie przepisów do ulubionych w swoim profilu użytkownika.
\end{itemize.zip}

\section{Odniesienie do innych rozwiązań}
\subsection{AniaGotuje.pl}
\begin{itemize.zip}
    \item Przeglądanie listy przepisów na stronie głownej zajmuje zbyt dużo czasu z powodu dużych zdjęć dań.
    \item Brak zwięzłej i krótkiej listy kroków przygotowania potrawy.
    \item Brak wyróżnionych sekcji ze składnikami i krokami przepisu.
    \item Przestarzały interfejs.
\end{itemize.zip}
\subsection{kwestiasmaku.com}
\begin{itemize.zip}
    \item Brak podstawowych informacji dotyczących dania (oceny użytkowników, czasu przygotowania, ilość porcji itd.) na stronie głównej.
    \item Źle zagospodorowana przestrzeń w widoku desktopowym.
\end{itemize.zip}
\section{Lista członków}
    Dominik Brzeziński, Kamil Kuziora, Bartosz Ludera, Viktoryia Manulenka
\end{document}
