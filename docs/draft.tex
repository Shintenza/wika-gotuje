\documentclass{article}
\usepackage{graphicx} % Required for inserting images
\usepackage{polski}
\usepackage{geometry}

\title{\fontsize{20}{22}\selectfont Projekt zespołowy\\ część 1\\Strona internetowa z przepisami kulinarnymi\\"Wika gotuje"}
\date{}

\begin{document}
\maketitle
\newpage
\tableofcontents
\newpage

\section{Opis problemu}
\noindent
"Wika gotuje" rozwiązuje problem związany z brakiem możliwości udostępniania swoich przepisów dla pasjonatów gotowania. Użytkownicy często napotykają trudności w znalezieniu prostych przepisów do gotowania, a także inspiracji kulinarnych, dzieleniu się swoimi pomysłami lub organizowaniu swoich przepisów. 
\newline
Istniejące platformy mogą zawierać zbyt wyszukane przepisy lub posiadać przestarzały, nieresponsywny interfejs.

\subsection{Cele użytkownika}
\begin{itemize}
    \item \textbf{Dzielenie się przepisami kulinarnymi: }Użytkownicy chcą dzielić się swoimi przepisami z innymi.
    \item \textbf{Odkrywanie nowych dań: }Poszukiwanie nowych, ekscytujących przepisów i kuchni, które zaspokoją ich gust.
\end{itemize}

\subsection{Przeszkody}
\begin{itemize}
    \item \textbf{Trudności w personalizacji}: Brak spersonalizowanych rekomendacji i filtrów, co sprawia, że użytkownicy tracą czas na przeszukiwanie wielu przepisów.
    \item \textbf{Skomplokowany interface}: Interfejs uniemożliwiający szybkie znalezienie cennych informacji (skłądniki, ilość porcji, stopień trudności, czas przygotowywania itd.)
\end{itemize}


\section{Używane technologie}
\begin{itemize}
    \item Next.js - frontend i backend framework,
    \item Sass - arkusze styli,
    \item Netlify - hosting,
    \item MongoDB - baza danych.
\end{itemize}

\section{Docelowy użytkownik}
"Wika gotuje" jest skierowana do wszystkich pasjonatów gotowania, niezależnie od poziomu doświadczenia kulinarnego. Różnorodność użytkowników obejmuje zarówno kuchcików-amatorów, jak i doświadczonych szefów kuchni. Platforma może być filtrowana pod względem preferencji diety i poziomów umiejętności, aby zapewnić spersonalizowane doświadczenia.


\section{Rozwiązanie}
Stworzenie aplikacji internetowej umożliwiającej:
\begin{itemize}
    \item łatwe utworzenie profilu użytkownika (z użyciem platform takich jak facebook, google itd.),
    \item dodawanie i ocenianie przepisów kulinarnych,
    \item możliwość pobierania przepisu w wersji PDF do druku,
    \item interface przyjazny do użytkowania w trakcie gotowania,
    \item filtrowanie i sortowanie przepisów po kategorii/poziomie trudności itd. ,
    \item dodawanie przepisów do ulubionych w swoim profilu użytkownika.
\end{itemize}

\section{Odniesienie do innych rozwiązań}
\subsection{AniaGotuje.pl}
\begin{itemize}
    \item Przeglądanie listy przepisów na stronie głownej zajmuje zbyt dużo czasu z powodu dużych zdjęć dań.
    \item Brak zwięzłej i krótkiej listy kroków do gotowania.
    \item Brak wyróżnionych sekcji z składnikami i krokami przepisu.
    \item Przestarzały interfejs.
\end{itemize}
\subsection{kwestiasmaku.com}
\begin{itemize}
    \item Brak informacji podstawowych dotyczących dania (brak oceny, czasu przygotowania, ilość porcji itd.).
    \item Źle zagospodorowana przestrzeń w widoku desktopowym.
\end{itemize}
\section{Lista członków}
\begin{itemize}
    \item Dominik Brzeziński
    \item Kamil Kuziora
    \item Bartosz Ludera
    \item Viktoryia Manulenka
\end{itemize}
\end{document}
