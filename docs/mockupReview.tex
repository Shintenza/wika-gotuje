\documentclass{article}
\usepackage{polski}
\usepackage{geometry}

\geometry{
    a4paper,
    total={190mm,270mm},
    left=10mm,
    top=10mm,
}

\begin{document}
\section{Sprawozdanie}
Celem projektu \textit{Wika Gotuje} jest stworzenie przyjaznej platformy do dzielenia się, odkrywania i angażowania się w różne przepisy kulinarne. Platforma zachęca użytkowników do udziału, umożliwiając im komentowanie, ocenianie i dodawanie własnych kulinarnych kreacji. Poniżej znajduje się kompleksowa recenzja oparta na dostarczonej opinii od recenzentów.

\section{Recenzja Projektu: Wika Gotuje}
\subsection{Widoczność Komentarzy po Kliknięciu na Przepis}
    Jedną z godnych uwagi cech projektu jest możliwość uzyskania dostępu do sekcji komentarzy poprzez kliknięcie na ikonę przepisu. Taki wybór projektowy wzmacnia interakcję użytkownika, dostarczając dedykowanego miejsca do dyskusji związanej z danym daniem. Ta funkcja skutecznie sprzyja tworzeniu społeczności i dzieleniu się wiedzą między użytkownikami.
\subsection{Wymagane Elementy Przepisu}
  Aby zapewnić spójną i informacyjną bazę danych, wprowadzenie wymogu dodawania zdjęć, nazwy dania i typu przepisu to godna pochwały decyzja. To pomaga utrzymać wizualnie atrakcyjne i dobrze zorganizowane repozytorium przepisów, ułatwiając użytkownikom nawigację i odnajdywanie konkretnych dań.
\subsection{Rozszerzone Opcje Wyszukiwania i Dodawania}
  Implementacja wielu opcji podczas zarówno wyszukiwania przepisów, jak i dodawania ich, dodaje głębi do doświadczenia użytkownika. Użytkownicy mogą teraz precyzować swoje wyszukiwania na podstawie określonych kryteriów, a autorzy przepisów mogą lepiej kategoryzować swoje dzieła. Ta funkcja usprawnia platformę, czyniąc ją bardziej przyjazną i efektywną.
\subsection{Potwierdzenie Usunięcia Konta}
    Dodanie potwierdzenia podczas procesu usuwania konta to przemyślana funkcja. Nie tylko zapobiega przypadkowym usunięciom konta, ale także sprawia, że użytkownicy są świadomi nieodwracalnego charakteru tego działania. To odzwierciedla zaangażowanie w doświadczenie użytkownika i bezpieczeństwo danych.
\subsection{Interakcje Wymagające Uwierzytelnienia}
    Decyzja o ograniczeniu funkcji komentowania i oceniania tylko do zalogowanych użytkowników dodaje warstwę odpowiedzialności do interakcji użytkowników. Taka strategia promuje poczucie społecznej odpowiedzialności i zniechęca do nadużyć czy spamowania. Ponadto, skłania użytkowników do aktywnego uczestnictwa poprzez tworzenie kont.
\subsection{Precyzyjne Informacje o Kraju Pochodzenia}
    Dodanie precyzyjnych informacji o kraju pochodzenia dania to cenny dodatek. To nie tylko wzmacnia aspekt edukacyjny platformy, ale także zaspokaja zainteresowania użytkowników o konkretnym charakterze kulinarystycznym. Dodaje to kontekstu kulturowego do przepisów, czyniąc platformę bogatszą i bardziej pouczającą.
\subsection{Podsumowanie}
    Projekt Wika Gotuje skutecznie uwzględni wartościową opinię, tworząc platformę dla pasjonatów gotowania. Dbanie o detale, projektowanie z myślą o użytkowniku i przemyślane funkcje przyczyniają się do pozytywnego doświadczenia użytkownika. Zaangażowanie w tworzenie społeczności pasjonatów gotowania jest widoczne, czyniąc platformę obiecującym miejscem do odkrywania i interakcji kulinarnej.
\end{document}